\documentclass{article}

\usepackage{titlesec}
\usepackage{titling}
\usepackage[margin=2cm]{geometry}
\usepackage{array}
\usepackage{tabularx}

\titlespacing{\subsubsection}{12pc}{1.5ex plus .1ex minus .2ex}{1pc}

\titleformat{\subsubsection}[wrap]
{\normalfont\bfseries}
{\thesection.}{0.5em}{}

\titleformat{\section}
{\Large\bfseries\uppercase}
{}
{0em}
{}[\titlerule]

\titleformat{\subsection}
{\large\bfseries}
{}
{0em}
{}

\titleformat{\subsubsection}
{\large\bfseries}
{}
{1em}
{}[]

\titlespacing{\subsection}
{0em}{1em}{0.3em}

%\titlespacing{\subsubsection}
%{0em}{0.5em}{0.1em}

\renewcommand{\maketitle}{
    \begin{center}
        \Huge\bfseries
        \theauthor

    \end{center}
    \begin{center}
        Bří Čapků 707, Krásno nad Bečvou, 757 01 Valašské Meziříčí
    \end{center}
}

\newcommand{\dsubsection}[2]{
    \subsection{\textbf{\large{#1}}\hfill\textmd{\small{#2}}}
}

\usepackage{lipsum}


\begin{document}

\title{CV}
\author{Albert Klinkovský}

\maketitle



\section{Projects}

\subsection{Interpreter}

\subsection{Lua compiler}


\subsection{Compiler for a subset of Lua language}

The goal of this project was to create a compiler for a statically typed subset of Lua language named Teal. The compiler
was written in pure C language. The compiler was able to compile a subset of Lua language to a bytecode which was then
interpreted by a virtual machine.

\subsection{Calculator parser / interpreter}

Goal of this project was to implement lexer, parser and interpreter for a mathematical calculator. The mathematical
expression is represented as a string such as "1 + 1" was converted to syntax tree data structure and then processed
by interpreter which returns the result.

\subsection{Acoustic localization system}

This is a final project at the school. The aim of this project was to locate clapping in space.  It was necessary to
connect various disciplines, such as:  mathematics, physics, electronics, software development. Most of the software
was written using Python and the Numpy, Scipy, PyAudio libraries.

\subsection{Signal generator}

This is a semester project, where we had the task to create a program for the ARM processor, which was to generate a
selected function using a DA converter.  The FreeRTOS operating system was used.


\section{Technology Summary}

{\renewcommand{\arraystretch}{1.5}%
\begin{tabularx}{\textwidth}{ >{\bfseries}l X }
    Programming       & Python (Numpy, SciPy), Lua, JavaScript, TypeScript, PHP, Bash, C, C++, Assembly, Rust, VHDL, HTML, XML, CSS, SASS, Markdown, Latex \\
    Software / Tools  & MS Office, Adobe CC, Blender, Ansible, Git (GitLab, GitHub, BitBucket), Trello, Grafana, Docker, UML \\
    Database          & MySQL, MariaDB, SQLite, PostgreSQL, InfluxDB, Redis, SQLAlchemy, Diesel (Rust) \\
    Frameworks        & Nette, Vue, React, Next.js, Django, Angular \\
\end{tabularx}}


\section{Work Experience}

\dsubsection{ERA a.s.}{September 2021 -- \textit{Present}}

I work with multiple teams using diverse languages and technologies. This has given me extensive knowledge and adaptability,
allowing me to collaborate effectively, communicate technical information clearly, and troubleshoot efficiently. I am a versatile and skilled
professional who delivers outstanding results.

\dsubsection{ON Semiconductor}{June 2021 -- September 2021}

I have been working as intern in international team (US). My job was to work on frontend for Self Service Portal using Angular framework.


\section{Education}

\dsubsection{VSB Technical University of Ostrava}{September 2022 -- \textit{Present}}

Faculty of Information Technology.

\dsubsection{Brno University of Technology}{September 2020 -- {June 2022}}

Faculty of Information Technology.

\dsubsection{Střední škola informatiky elektrotechniky a řemesel RpR.}{September 2015 -- June 2020}

Field of study is electronic data processing.

\end{document}